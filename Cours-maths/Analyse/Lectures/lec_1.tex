\chapter{Primitives et équations differentielles}
\section{Equations differentielles de type : \(y' = f\)}

\begin{definition}[Primitive]\label{def:primitive}
    Soit : \begin{align*}
        F: I &\longrightarrow \mathbb{R} \\
         x &\longmapsto F(x) 
    \end{align*}
    \(F\) est une primitive de \(f\) ssi \(F'(x) = f(x)\). On note \(F' = f\) 
\end{definition}

\begin{theorem}[Primitive d'une fonction continue]\label{thm:pcont}
    Si \(f\) est continue, alors l'équation \(y' = f\)  admet au moins une solution.
\end{theorem}

\begin{theorem}[Unicité de la primitive]\label{thm:uniprim}
    Deux primitives d'une même fonction sont égales à une constante près. 
\end{theorem}

\begin{explanation}
    Soient \(F_{1}\) et \(F_{2}\), deux primitives de f. On a : \(F_{1}' = f\) et \(F_{2}' = f\). D'ou, par différence de fonctions dérivables, \(F_{1}-F_{2}\) est dérivable.\\
    \((F_{1}-F_{2})' = F_{1}'-F_{2}' = f-f = 0\) D'ou on a : \(F_{1}-F_{2} = a, a \in \mathbb{R}\)      
\end{explanation}
\begin{corollary}[Propriété : unicité de la solution]\label{col:unisol}
    Si \(x_{0} \in I \subset \mathbb{R}\) et \(F(x_{0}) = y_{0}\), alors on a : \\
    \[
        \exists! F: x \mapsto F(x), F' = f
    \]  
\end{corollary}

\begin{explanation}
    Supposons : \(\exists (F;G) \in \mathcal{F}(\mathbb{R},\mathbb{R})^{2} \text{ t.q. }\) \(\begin{cases}
        &F' = G' = f\\
        &F(x_{0}) = G(x_{0}) = y_{0}\\
        &F \neq G
    \end{cases}\)\\
    Donc, d'après \autoref{thm:uniprim} \(\exists a \in \mathbb{R}^{*}, F(x) = G(x) + a\) 
    \begin{align*}
        F(x_{0}) = G(x_{0}) & \iff y_{0} = y_{0} + a\\
        & \iff a = 0 \text{ absurde!}
    \end{align*} 
    
    Donc \(F=G\) 
\end{explanation}
\section{Recherche de primitives}
Pour identifier une primitive de \(f\) on peut : 
\begin{enumerate}
    \item Reconnaitre la dérivée d'une fonction de réference
    \item Adapter éventuellement le coefficient
    \item Reconnaitre une des formules suivantes : \begin{table}[H]
        \centering
        \begin{tabular}{c|c}
            \toprule
                Fonction reconnue &  Primitive \\
            \midrule
                \(\frac{u'}{u}\)  &  \(\ln \lvert u \rvert \)  \\
                \(u'u\)  &  \(\frac{1}{2}u^{2}\)  \\
                \(u'u^{n}\) &  \(\frac{u^{n+1}}{n+1}\)  \\
                \(u' \cdot v' \circ u\) & \(v \circ u\)  \\
            \bottomrule
        \end{tabular}
        \caption{Formules de primitives}
        \label{tab:primitives}
    \end{table}
\end{enumerate}
\begin{corollary}[Propriétés de la primitive]\label{col:propprim}
    \begin{itemize}
        \item Si \(F\) est une primitve de \(f\), alors \(aF\) est une primitive de \(af\)
        \item Si \(F\) et \(G\) sont des primitives de \(f\) et \(g\) respectivement, alors \(F+G\) est une primitive de \(f+g\)  
    \end{itemize}
\end{corollary}
\begin{eg}[Exemples]\label{eg:primitives}
    \begin{itemize}
        \item \(f_{1}(x) =  \frac{7x}{x^{2}+1} \implies F_{1}(x) = \frac{7}{2} \ln \lvert x^{2}+1 \rvert \) 
        \item \(f_{2}(x) = 2x(x^{2}+3) \implies F_{2}(x) = \frac{1}{2}(x^{2}+3)^{2}\) 
        \item \(f_{3}(x) = \cos x \sin^{3}x \implies F_{3}(x) = \frac{1}{4}\sin^{4}(x)\) 
        \item \(f_{4}(x) = x^{3}\sqrt{x} = x^{\frac{7}{2}}, \implies F_{4}(x) = \frac{2}{9}x^{\frac{9}{2}} = \frac{2}{9}x^{4}\sqrt{x}\)
    \end{itemize}
\end{eg}

\section{Equations differentielles de la forme \(y' = ay+f\) ou \(y' = ay+f\)}
\subsection{Equation de la forme \(y' = ay\) }
L'équation différentielle \(y' = ay\) est une equation différentielle du premier ordre linéaire et homogène à coefficient constant. Cette équation \((E_{0})\) s'écrit également souvent \(y'-ay = 0\)
\begin{corollary}[Propriétés]\label{col:sol_1}
    Les solutions de \((E_0)\) sont de la forme : 
    \[
        \lambda e^{ax}, C \in \mathbb{R}
    \] 
    Si on ajoute la condition initiale \(y(x_{0}) = y_{0}\), on a une solution unique.
\end{corollary} 

\begin{corollary}[Propriété]\label{col:homo}
    Si \((E_{0})\) a deux solutions \(y_{1}\) et \(y_{2}\) : \(\forall k \in \mathbb{R}\) \(ky_{1}\) est solution et \(y_{1}+y_{2}\) est solution.      
\end{corollary}
\subsection{Equation de la forme \(y' = ay + b\) }

L'équation différentielle \(y' = ay+b\) est appelée équation differentielle linéaire du premier ordre à coefficient constant avec second membre constant. Elle s'écrit également \(y'-ay = b\).\\ 
\begin{corollary}[Propriétés]\label{col:sol_2}
    Les solutions sont de la forme : 
    \[
        y = \lambda e^{ax} -\frac{b}{a}
    \]
\end{corollary}

\begin{eg}[Exemple]\label{eg:resol_1}
    Résoudre l'équation différentielle \((E) \iff -2y' = 7y+6\).\\
    On a : \((E) \iff y' = -\frac{7}{2}y -3\). On reconnait une équation de la forme \(y' = ay+b \) avec \(a=-\frac{7}{2}\) et \(b = -3\). \(\frac{b}{a} = \frac{6}{7}\). Les solutions sont donc de la forme \(y(x) = \lambda e^{-\frac{7}{2}x}+ \frac{6}{7}, \lambda \in \mathbb{R}\)
\end{eg}