\chapter{Loi des grands nombres}
\section{L'inégalité de Bienaymé-Tchebychev}
\subsection{L'inégalité de Markov}

\begin{definition}[Variable aléatoire positive ou nulle]
    Une variable aléatoire est dite \textit{positive ou nulle} dans un univers \(\Omega\), lorsque toutes les valeurs prises par celles-ci sont des réels positifs ou nuls 
\end{definition}

\begin{theorem}[L'inégalité de Markov]
    Soit \(X\) une variable aléatoire réelle positive ou nulle d'espérance \(E(X)\). Alors, pour tout réel \(a\) strictement positif, \(P(X\geq a)\leq \frac{E(X)}{a}\). Cette inégalité est appelée \textit{Inégalité de Markov}.
\end{theorem}

\subsection{L'inégalité de Bienaymé-Tchebychev}

\begin{theorem}[Inégalité de Bienaymé-Tchebychev]
    Soit \(X\) une variable aléatoire d'espérance \(E(X)\) et de variance \(V(X)\). Alors, pour tout réel \(a\) strictement positif, \(P(\lvert X-E(X) \rvert \geq a ) \leq \frac{V(X)}{a^{2}}\). Cette inégalité est appelée \textit{Bienaymé-Tchebychev} 
\end{theorem}

\begin{corollary}[Propriété]
    On peut réécrire l'inégalité sous la forme suivante : 
    \[
        P(\lvert X-E(X) \rvert < a) \geq 1- \frac{V(X)}{a^{2}}
    \]
\end{corollary}

\section{Loi des grands nombres}
\subsection{L'inégalité de concentration}

\begin{theorem}[L'inégalité de concentration]
    Soit \(X\) une variable aléatoire. On pose \(M_{n}\) la variable aléatoire moyenne d'un échantillon de taille \(n\) de \(X\), \(M_{n} = \frac{1}{n} \sum_{i=1}^{n}X_{i}\) où les variables \(X_{i}\) sont indépendantes et de même loi que \(X\).\par
    Alors, on a : 
    \[
        P(\lvert M_{n} - E(X) \rvert \geq a ) \leq \frac{V(X)}{na^{2}}
    \]   
    Cette inégalité est appelée \textit{inégalité de concentration}.
\end{theorem} 

\newpage
\subsection{Loi faible des grands nombres}

\begin{theorem}[Loi faible des grands nombres]
    Soit \((X_{n})\) un échantillon d'une variable aléatoire. On pose \(M_{n} = \frac{1}{n} \sum_{i=1}^{n} X_{i}\). Alors, pour tout réel \(a\) strictement positif, on a : 
    \[
        \lim_{n \to \infty}P(\lvert M_{n} - E(X) \rvert \geq a ) = 0
    \]  
\end{theorem}

