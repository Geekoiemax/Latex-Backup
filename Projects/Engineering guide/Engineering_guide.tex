\documentclass[a4paper]{report}
\input{header.tex}
\author{Maxime Muller}
\title{STEM Racing : Engineering your future}

\thispagestyle{empty}

\usepackage{adjustbox}
\usepackage{centernot}
\usepackage{lmodern}
\graphicspath{{./Image/}}

\begin{document}

\maketitle

\begin{abstract}
The STEM Racing challenge, formerly F1 in Schools, is the world's largest secondary school technology program, engaging millions of students globally in a hands-on application of Science, Technology, Engineering, and Mathematics (STEM) principles. Teams design, manufacture, and race miniature CO2-powered Formula 1 cars, aiming to introduce young individuals to engineering in an engaging environment. This guide illuminates the path for aspiring engineers, outlining their critical role, available tools, and essential questions that drive innovation. \par

In STEM Racing, the engineer is central to success, overseeing design, analysis, manufacturing, and documentation, while ensuring compliance with technical regulations. This role demands balancing creative problem-solving with adherence to constraints and effectively communicating design choices. \par

This guide will illuminate pathways to essential resources for your engineering journey, helping you understand the types of tools available for design, simulation, and manufacturing, such as computer-aided design (CAD), computer-aided engineering (CAE), and computer-aided manufacturing (CAM) processes. \par

Rather than providing answers, this guide fosters innovation by prompting fundamental engineering questions. It encourages critical examination of design choices, materials, and manufacturing, guiding you to inquire: "How does the car's profile influence drag?" or "What material properties balance strength and weight?" This approach cultivates a problem-solving mindset, empowering independent research and discovery. \par

This guide serves as a foundational starting point for your STEM Racing engineering journey, providing initial direction and critical questions, but it is by no means an exhaustive manual for guaranteed success. Engineering demands continuous learning, experimentation, and adaptability. Success hinges on collaborative spirit, iterative design, hands-on application, and learning from both triumphs and setbacks. This guide is not a magic bullet, it serves not to walk you through the entirety of the competition but only to get you started. Success hinges not on reading this guide but on research, innovation, and mastery of the tools and the regulations.. 

\end{abstract}

\newpage

\tableofcontents

\lec{1}{3}

\end{document}