\lecture{2}{24 Mai\ 10:00}{Session 2}
\section{Le principe de relativité de Galilée}

\begin{definition}[Système de référence]
    Pour étudier les phénomènes mécaniques, il faut choisir un \textit{un système de référence}. Un système de réference et dit galiléen si l'espace est homogène et isotrope et que le temps est uniforme.
\end{definition}

\begin{theorem}[Principe d'inertie]
    Etduions une particule se déplaçant librement dans un réferentiel galiléen. Dans un réferentiel galiléen, le temps et l'espace sont uniforme, donc \(L\) ne dépend ni de \(\vec{r}\) ni de \(t\). De plus, l'espace est aussi isotrope, d'où \(L\) ne dépend pas de la direction de la vitesse, d'où : 
    \begin{equation}
        L = L(v^{2})
    \end{equation}  
\end{theorem}
\newpage
\begin{corollary}[Principe d'inertie]
    On a alors :
    \[
        \frac{\partial L}{\partial r} = 0 \implies  \frac{d}{dt}\frac{\partial L}{\partial v} = 0 \implies \frac{\partial L}{\partial v} = \text{ cste}.
    \]
    Mais \(L = L(v^{2})\), donc on en déduit que \(v = \text{ cste } \). \par
    Donc dans un réferentiel galiléen, tout mouvement libre s'effectue avec une vitesse constante : c'est le \textit{principe d'inertie}. 
\end{corollary}