\documentclass{article}
\usepackage[utf8]{inputenc}
\usepackage{amsmath,amssymb,amsfonts} %pour les maths
\usepackage{tikz,ifthen,tikz-3dplot} %outil pour les shcemas 
\usepackage[]{wasysym} %horloges

%GEOMETRY
\usepackage[a4paper,top=1in,bottom=1in,left=1in,right=1in,heightrounded]{geometry}
%line height
\renewcommand{\baselinestretch}{1.15}
%paragraphs
\setlength{\parindent}{0pt}
\setlength{\parskip}{0.8em}


\title{Cinématique Relativiste}
\author{Maxime Muller}
\date{\today}


\begin{document}
\maketitle

Ceci est une transcription des cours de cinématique relativiste donné par le professeur Xavier Ovido au lycée St Louis de Gonzague dans le cadre du club Phythème.

\section{La fin de l'age classique}
A la fin du 19e siècle et au début du 20e, on trouve trois grandes théories.
\begin{itemize}
    \item La mécanique de Newton
    \item L'électromagnétisme de Maxwell (électricité + magnétisme + optique)
    \item La thermodynamique
\end{itemize}
\subsubsection{Maxwell}
Prédiction d'ondes électromagnétiques qui ont les même propriétés que la lumière et ont une vitese de propagation :
\begin{equation*}
    c = \frac{1}{\sqrt{\varepsilon_0 \mu_0}}
\end{equation*}
Avec :
\begin{itemize}
    \item $\varepsilon_0$ : permittivité du vide
    \item $\mu_0$ : perméabilité du vide
\end{itemize}
\subsubsection{Thermodynamique}
Elle s'interesse aux machines thermiques et plus généralement aux échanges d'énergie.
\newpage
\subsection{La vielle mécanique de Newton}
\subsubsection{La première loi de Newton}
\begin{itemize}
    \item Aucun réferentiel absolu
    \item Pas de vitesse absolue
\end{itemize}
\subsubsection{La deuxième loi de Newton}
Dans un réferentiel galiléen, on a :
\begin{equation*}
    \vec{f} = m \cdot \frac{d\vec{v}}{dt}
\end{equation*}

La mécanique est basée sur un postulat fondamental, qui est le principe de relativité :
Les lois de la mécanique sont les mêmes dans tous les réferentiels galiléen.
\\
\\

%inserer premier schema (see tikz 3d plot)
\tdplotsetmaincoords{0}{0}

\begin{tikzpicture}[tdplot_main_coords]

% Create a 3D coordinate system
\draw[thick,->] (0,0,0) -- (7,0,0) node[anchor=north east]{$x$};
\draw[thick,->] (0,0,0) -- (0,5,0) node[anchor=north west]{$z$};
\draw[thick,->] (0,0,0) -- (-2.5,-2.5,5) node[anchor=south]{$y$};
\draw[thick,->] (2.5,0,0) -- (2.5,5,0) node[anchor=north west]{$z'$};
\draw[thick,->] (2.5,0,0) -- (0,-2.5,5) node[anchor=south]{$y'$};
\draw[thick,color=gray,->] (2.5,2.5,0) -- (4,2.5,0) node[anchor=north east]{$\vec{v}_e$};
\node[anchor=north] at (0.2,4.7,0) {\clock}; % Clock symbol
\node[anchor=north] at (0.5,4.7,0) {$H$}; % Label H next to the clock
\node[anchor=north] at (0.2+2.5,4.6,0) {\clock}; % Clock symbol
\node[anchor=north] at (0.5+2.5,4.6,0) {$H'$}; % Label H next to the clock

\end{tikzpicture}

\underline{Cinématique relativiste :} \\
\begin{equation*}
    \begin{cases}
        t' = t         \\
        x' = x - v_e t \\
        y' = y         \\
        z' = z
    \end{cases}
    \Leftrightarrow
    \boxed{
        \begin{cases}
            t' = t \\
            \vec{r'} = \vec{r} - \vec{v}_{e} t
        \end{cases}
    }
\end{equation*}
\newpage
D'après le PFD :

\begin{align*}
    \frac{d^2\vec{r}}{dt^2}   & = \frac{1}{m} \cdot \vec{f}               \\
    \intertext{et de même :}                                              \\
    \frac{1}{m} \cdot \vec{f} & = \frac{d^2\vec{r'}}{dt^2}                \\
                              & = \frac{d^2}{dt^2}(\vec{r} - \vec{v}_e t) \\
                              & = \frac{d^2\vec{r}}{dt^2}                 \\
                              & = \frac{1}{m} \cdot \vec{f}
\end{align*}

La résultante des forces $\vec{f}$ est la même dans les deux réferentiels.

\subsubsection{La troisième loi de Newton}
\begin{equation*}
    \vec{f}_{i\leftarrow j} = - \vec{f}_{j\leftarrow i}
\end{equation*}
La vitesse de propagation des interactions est infinie.\\


\subsubsection{Pourquoi la relativité restreinte?}
L'électromagnétisme conduit à l'existence d'onde de vitesse $c = \frac{1}{\sqrt{\varepsilon_0 \mu_0}}$.\\

De plus, une particule électromagnétique subit une force qui est la force de Lorentz :
\begin{align*}
    \vec{f} & = q \cdot (\vec{E} + \vec{v} \wedge \vec{B})       \\
            & = q \cdot \vec{E} + q \cdot \vec{v} \wedge \vec{B} \\
            & = \vec{F_e} + \vec{F_m}
\end{align*}

Calculons $f'$ :
\begin{align*}
    \vec{f'} & = q \cdot (\vec{E'} + \vec{v'} \wedge \vec{B'})  \\
    \vec{f'} & = q[\vec{E'}+ (\vec{v}-\vec{v_e})\wedge \vec{B}]
\end{align*}
Par identification : 
\begin{center}
    \boxed{
        \begin{aligned}
            \vec{B'} & = \vec{B}                          \\
            \vec{E'} & = \vec{E} + \vec{v_e}\wedge\vec{B}
        \end{aligned}
    }
\end{center}

La force de Lorenz n'est pas un invariant par transformation de Galilée.\\
Comment résoudre le problem?
\begin{enumerate}
    \item Admettre que la théorie de Maxwell est fausse (difficile, car elle explique et prédit plusiers phénomènes physiques)
    \item Rendre les particules de la mécanique classique et d'électromagnétisme compatibles en ajoutant des principes qui n'ont pas lieu d'être
    \item Admettre que les postulats de la mécanique classique sont faux
\end{enumerate}
\section{Postulats de la Relativité restreinte}
\subsection{Premier postulat}
Les lois de la physique ont la même forme dans tous les réferentiels d'inertie.
\begin{equation*}
    \Phi(x,y,z,t) = f(x,y,z,t) \Rightarrow \Phi'(x',y',z',t') = f(x',y',z',t')
\end{equation*}
\subsection{Deuxième postulat}
La vitesse de la lumière est la même dans tous les réferentiels d'inertie. On y ajoute le principe d'équivalence :\\
Pour $v<<c$ on retrouve les lois de Newton et les transformations de Galilée.\\
\section{Evenements et transformations spéciales de Lorentz} 
\subsection{Evenements}
\underline{Def} : on appelle évènement un phénomène $\phi$ supposé infiniment localisé dans l'espace et le temps
\subsubsection{Coïncidence}
Deux évènements sont coincidents s'ils ont lieu au même endroit et au même moment. La coïncidence de deux évènements est un fait absolu, indépendant du réferentiel.
\subsubsection{Horloges Synchrones}

%Faire le schema, voire carnet Tachyon
\tdplotsetmaincoords{0}{0}

\begin{tikzpicture}[tdplot_main_coords]

% Create a 3D coordinate system
\draw[thick,->] (0,0,0) -- (7,0,0) node[anchor=north east]{$x$};
\draw[thick,->] (0,0,0) -- (0,5,0) node[anchor=north west]{$z$};
\node[anchor=north] at (0.2,4.7,0) {\clock}; % Clock symbol
\node[anchor=north] at (0.5,4.7,0) {$H$}; % Label H next to the clock
\node[anchor=north] at (0.2+2.5,4.6,0) {\clock}; % Clock symbol
\node[anchor=north] at (0.5+2.5,4.6,0) {$H'$}; % Label H next to the clock
\draw[thick,->] (0,0,0) -- (-2.5,-2.5,5) node[anchor=south]{$y$};
\draw[thick,->] (2.5,0,0) -- (2.5,5,0) node[anchor=north west]{$z'$};
\draw[thick,->] (2.5,0,0) -- (0,-2.5,5) node[anchor=south]{$y'$};
\draw[thick,color=gray,->] (2.5,2.5,0) -- (4,2.5,0) node[anchor=north east]{$\vec{v}$};

\end{tikzpicture}


Si pour tout $t_a$ on a $t_b = t_a + \frac{AB}{c}$, alors les horloges sont synchrones.
Il s'agit d'un protocole pour synchroniser toutes les horloges dans un réferentiel donné.
\subsubsection{Evenements simultanés}
Deux évenements sont simultanés s'ils ont lieux au même moment selon les horloges locales de A et de B préalablement sychronisées.\\
La simultanéité n'est pas un invariant par changement de réferentiel.
\subsubsection{Intervalle entre deux évènements}
%Schema, voire carnet Tachyon
%compléter sur qqun d'autre

Soit : $(\Delta s)^2 = c^2 (t-t')^2 - [(x-x')^2 + (y-y')^2 + (z-z')^2]$
Si : 
\begin{itemize}
    \item $(\Delta s)^2 = 0 \Rightarrow \text{Sur le cone de lumière}$
    \item $(\Delta s)^2 >0 \Rightarrow \text{intevalle de temps}$
    \item $(\Delta s)^2 > 0 \Rightarrow \text{intervalle de temps}$
\end{itemize}

On peut montrer que $(\Delta s)^2$ est un invariant scalaire par changement de réferentiel.
\subsection{Transformations de Lorentz (1904)}
\subsubsection{Matrices de transformation}
\begin{equation*}
    \begin{pmatrix}
        ct\\
        x\\
        y\\
        z
    \end{pmatrix}
     = \begin{pmatrix}
        \gamma & \beta\gamma&0&0\\
        \beta\gamma & \gamma &0&0\\
        0&0&1&0\\
        0&0&0&1\\
     \end{pmatrix}
     \begin{pmatrix}
        ct'\\
        x'\\
        y'\\
        z'
    \end{pmatrix}
\end{equation*}
\linebreak
\linebreak
$\text{Avec : } \gamma = \frac{1}{\sqrt{1-\beta^2}} \text{ et } \beta = \frac{v}{c}$
Pour passer du réferentiel R' à R, il suffit de remplacer $v$ par $-v$
\begin{equation*}
    \begin{pmatrix}
        ct'\\
        x'\\
    \end{pmatrix}
     = \begin{pmatrix}
        \gamma&-\beta\gamma\\
        -\beta\gamma&\gamma
     \end{pmatrix}
     \begin{pmatrix}
        ct\\
        x
     \end{pmatrix}
\end{equation*}
\(
    v<<c \Rightarrow \beta<<1 \text{ et } \gamma \rightarrow 1\\
    \begin{cases}
        ct' &= \gamma(ct - \beta x)\\
        &\approx ct\\
        t'&=t
    \end{cases}\\
    x' = \gamma(x - \beta ct) \approx x-vt
\)\\
$\Rightarrow$ On retrouve les transformations de Galilée
\subsubsection{Transformations des vitesses}
On a : $v'_x = \frac{dx'}{dt'}$\\
Or : \\
\[
\begin{cases}
    dx' &= \gamma(dx - \beta c dt)\\
    cdt' &= \gamma(cdt - \beta dx)
\end{cases}
\]
\begin{eqnarray*}
    \Rightarrow \frac{1}{c} \frac{dx'}{dt'} &=& \frac{\gamma(dx - \beta c dt)}{\gamma(cdt - \beta dx)}\\
    &=& \frac{dt(\frac{dx}{dt} - v)}{cdt (1- \frac{v}{c^2}\frac{dx}{dt})}\\
    &=& \frac{1}{c} \frac{v_x - v}{1 - \frac{v_x \cdot v}{c^2}}\\
    \Rightarrow v'_x&=& \frac{v_x - v}{1 - \frac{v_x \cdot v}{c^2}}
\end{eqnarray*}
Si $v_x=c$
\begin{equation*}
    v'_x = \frac{c-v}{1- \frac{c-v}{c^2}} = (c-v) \cdot \frac{c^2}{c^2 - cv} = \frac{c^2}{c} = c
\end{equation*}
Donc la vitesse de la lumière est un invariant par rapport au réferentiel.\\
Si $v_x << c $\\
alors $\frac{v_x}{c}<<1$, on trouve $v'x \approx v_x -v$ (Le terme $\frac{v_x\cdot v}{c^2}$ est négligeable dans la formule)\\
$\rightarrow$ On retrouve la loi classique de transformation.
\section{Temps propre et dilatation des durées}
\underline{Def} : L'intervalle de temps propre est l'intervalle de temps séparant deux horloges dans le réferentiel R' dans lequelle les horloges sont aux repos.
\begin{equation*}
\end{equation*}
\begin{align*}
    &ct = \gamma(ct'+\beta x')\\
    \intertext{Si H est au Repos dans R' :}\\
    &\Rightarrow \Delta x' = 0\\
    &\Rightarrow c\Delta t = c\gamma\Delta t'\\
    &\Rightarrow \Delta t = \gamma \Delta t'\\
    &\text{Or : } \gamma = \frac{1}{\sqrt{1-\beta^2}} \geq 1
\end{align*}
$\Rightarrow$ Dans R, il y a dilatation des durées par rapport au réferentiel propre R\\
\underline{Remarque} : On peut le montrer aussi avec $(\Delta s)^2$ \\
\begin{align*}
    ds^2 &= c^2 d\tau^2 \text{       (où $d\tau$ est le temps propre)}\\
    &= c^2 dt^2 -(dx^2 +dy^2 + dz^2)\\
    &= c^2 dt^2 (1-\frac{dx^2+dy^2+dz^2}{c^2dt^2})\\
    &= c^2 dt^2(1-\frac{1}{c^2}(\frac{d\vec{r}}{dt})^2)\\
    &=c^2dt^2 (1-\beta^2)\\
    &\Rightarrow d\tau^2 = \frac{dt^2}{\gamma^2} \Leftrightarrow dt = \gamma d\tau
\end{align*}

\section{Longueurs propres et contraction des longueurs}
\underline{Définition} : La longueur propre d'un objet est égale à la longueur de l'objet mesuré dans le réferentiel où l'objet est immobile.\\
Pour l'observateur dans R :\\
\[
\begin{cases}
    x_2' = \gamma(x_2 - \beta ct) \\
    x_1' = \gamma(x_1 - \beta ct)
\end{cases}
\]
\begin{equation*}
    \Rightarrow L' = x_2' - x_1' = \gamma (x_2-x_1) = \gamma L
\end{equation*}
Dans R, il y a une contrastion des longueurs (car $\gamma \geq 1$)\\
\underline{Conséquences} : \\
Les volumes subissent eux aussi une contraction : \\
$dV_0 = dx_0dy_0dz_0$ est le volume propre\\
Pour un mouvement selon $O_x$ : 
\begin{equation*}
   dV = dxdydz = \frac{dx_0}{\gamma}dy_0dz_0 = \sqrt{1-\beta^2}dV_0
\end{equation*}
Or la charge est un invariant relativiste, on a donc : 
\begin{equation*}
    \rho = \frac{dq}{dV} = \gamma \frac{dq}{dV_0} = \gamma \rho_0
\end{equation*}

\section{Notion de quadrivecteur}
\underline{Définition} : 4 quantités $A_1;A_2;A_3;A_4$ sont les composantes d'un quadrivecteur $\tilde{A}$ si vis à vis d'un changement de réferentiel galiléen, elles de tranforment de la même manière que les composantes du quadrivecteur $\tilde{OM} = (ct;\vec{r})$ \\
Si R et R' sont dans les conditions de la transformation de Lorenz, la transformation de $\tilde{A}$ est régi par la matrice de Lorenz.\\

\underline{Définition} : On définit la pseudonorme du 4-vecteur $\tilde{A}$ comme suit:
\[
\text{Soit } \tilde{A}\begin{pmatrix} A_0\\A_1\\A_2\\A_3 \end{pmatrix}, \space  \tilde{A} \cdot \tilde{A} = A_0^2 -(A_1^2 +A_2^2 +A_3^2)    
.\] 
On peut avoir $\tilde{A} \cdot \tilde{A}$ positif ou négatif. \\
\underline{Exemple} : Dans (R): $\tilde{OM} : (ct;\vec{OM})$ et $dt = \gamma d\tau$ avec $\gamma = \frac{1}{\sqrt{1-\frac{v^2}{c^2}} }$.\\
De façon évidente : $\tilde{v} = \frac{d\tilde{OM}}{d\tau}$\\
\[
\tilde{v} = \frac{d\tilde{OM}}{d\tau} = \frac{d\tilde{OM}}{dt} \cdot \frac{dt}{d\tau}
.\] 
Or : $\frac{dt}{d\tau} = \gamma $ et $d\tilde{OM} = (cdt;d\vec{OM})$
\[
\implies \tilde{v} = \gamma (c\space;\frac{d\vec{OM}}{dt}) \text{ et } \tilde{v} \cdot \tilde{v} = \gamma (c^2 -v^2) = \frac{c^2 }{c^2 -v^2 } (c^2 -v^2) = c^2 \text{ Invariant } \square 
.\] 



\end{document}