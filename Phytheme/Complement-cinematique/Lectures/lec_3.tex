\subsection{Petites oscillations au voisinage d'une position d'équilibre stable}
\subsubsection{Approche générale d'oscillateur harmonique}

Soit un système unidimensionnel caractérisé par une variable \(x\), soumis à un ensemble de forces résultantes \(\vv{f}\). 
\[
    \vv{f} = f_{x}\vv{i} , \, f(x) = -\frac{dE_{p}}{dx}
\] 

Soit \(x_{\text{0}}\) la position d'équilibre stable et \(k = \frac{d^{2}E_{p}}{dx^{2}}(x_{\text{0}})\).
\[
    E_{p}(x) \approx E_{p}(x_{\text{0}}) + (x-x_{\text{0}})\frac{dE_{p}}{dx}(x_{\text{0}}) + \frac{1}{2}(x-x_{\text{0}})^{2}k 
\] 
\[
    E_{p}(x) \approx E_{p}(x_{\text{0}}) + \frac{1}{2}k(x-x_{\text{0}})^{2}
\]
\[
    \vv{f} = -\frac{dE_{p}}{dx} \vv{i} = -k(x-x_{\text{0}})\vv{i}
\]

Autour d'une position d'équilibre stable, si l'écart à l'équilibre n'est pas trop important, le système subit \textbf{une force de rappelle elastique linéaire}, c-à-d propotionelle à l'écart.

L'équation du mouvement peut s'écrire : 
\[
    m \frac{d^{2}x}{dx^{2}} = -k(x-x_{\text{0}})
\]
On pose : \(X = x-x_{\text{0}}\), donc \(\frac{d^{2}X}{dx} = \frac{d^{2}x}{dx^{2}}\). Et on pose \(\frac{k}{m } = \omega_{\text{0}}^{2}\) 
\[
    \frac{d^{2}X}{dx^{2}} + \omega_{\text{0}}^{2}  = 0 
\]

\begin{theorem}[Résolution d'équations différentielles du second ordre]\label{thm:ED2SSM}
        On considère une ED du second ordre, SSM, à coeficients constants : 
        \[
            ay'' + by' + cy = 0 \, , a \neq 0
        \]
        A cette ED, on associe le polynome caractéristique : \\
        \[
            ar^{2}+br+c = 0 \implies \Delta = b^{2}-4ac
        \]
        3 cas :\\
        \begin{itemize}
            \item Cas 1 : \(\Delta >0\)\\
                \(\exists, r_{1}, r_{2} \in \mathbb{R}\) solutions de léquations.\\
                \[
                    \begin{cases}
                        r_{1} = \frac{-b + \sqrt{\Delta}}{2a}\\
                        r_{2} = \frac{-b-\sqrt{\Delta}}{2a}
                    \end{cases}
                \]
                La solution de l'ED est : \(y = A e^{r_{1}x} + B e^{r_{2}x}\), avec \(A\) et \(B\) à déterminer. 
            \item Cas 2 : \(\Delta = 0\)\\
                \(\exists! r\) solution : \(r= -\frac{b}{2a}\).\\
                La solution de l'ED est : \(y = \left( A'x+B' \right)e^{rx}\) avec \(A'\) et \(B'\) à déterminer.
            \item Cas 3 : \(\Delta <0\) \\
                \(\exists r_{3}, r_{4} \in \mathbb{C}\) : \\
                \[
                    \begin{cases}
                        r_{3} = \frac{-b-j\sqrt{-\Delta }}{2a}\\
                        r_{4} = \frac{-b+j\sqrt{-\Delta}}{2a}
                    \end{cases} \text{ avec } j^{2} = -1
                \] 
                On a alors : 
                \begin{eqnarray*}
                    y &=& A'' e^{r_{3}x}+ B''e^{r_{4}x}\\
                    &=& \left( A'' \exp \left[ \frac{-j-\sqrt{-\Delta}}{2a} \right] + B''e^{\frac{-j+\sqrt{-\Delta }}{2a}} \right)e^{-\frac{b}{2a}x}\\
                    &=& \left( \alpha \cos (\frac{\sqrt{-\Delta}}{2a}x) + \beta \sin (\frac{\sqrt{-\Delta }}{2a}x ) \right)e^{-\frac{b}{2a}x}
                \end{eqnarray*}
        \end{itemize} 
\end{theorem}

On a donc ici une ED du second ordre, sans second membre : 
\[
    \frac{d^{2}x}{dx^{2}} + \omega_{\text{0}}^{2}x = 0
\]

On y associe donc le polynome caracteristique : 
\[
    r^{2} + \omega_{\text{0}}^{2} = 0, \Delta = -4\omega_{\text{0}}^{2}
\]

On a donc deux racines complexes et la solution est donc : 
\[
    \boxed{x = A\cos\left( \omega_{\text{0}}t + \phi \right)}
\]

\begin{eg}[Exemple du pendule]
    \(\ddot{\theta} + \frac{g}{l}\sin \theta = 0\). Mais si \(\theta (\text{ en } \unit{rad}) \ll 1\), \(\sin \theta \approx \theta\). D'où : 
    \begin{eqnarray*}
        \ddot{\theta} + \frac{g}{l}\theta &=& 0 \\
        \implies \ddot{\theta} \omega_{\text{0}}^{2}\theta &=& 0, \text{ avec } \omega_{\text{0}}^{2} = \frac{g}{l} \\
        \implies \theta &=& A \cos\left( \omega_{\text{0}}t + \varphi \right)
    \end{eqnarray*}
    On utilise les conditions initiales pour déterminer les valeurs des constantes (\(A \text{ et } \varphi\)) : 
    \[
        \begin{cases}
            \theta(0) = \theta_{\text{0}} \\
            \dot{\theta}(0) = 0
        \end{cases}
    \]
    On a donc : 
    \begin{eqnarray*}
        \theta(0) &=& A \cos \left( \varphi \right) \\
        \dot{\theta}(t) &=& -A \omega_{\text{0}}\sin \left( \omega t +\varphi \right)\\
        \implies \dot{\theta}(0) &=& -A\omega_{\text{0}}\sin \left( \varphi \right) \\
        \implies \sin \varphi &=& 0 \\
        \implies \varphi &=& 0 \\
        \implies A\cos 0 &=& \theta_{\text{0}} \\
        \implies A &=& \theta_{\text{0}}
    \end{eqnarray*}
    \[
        \boxed{\theta(t) = \theta_{\text{0}} \cos(\omega_{\text{0}} t)}
    \]
    Or \(\omega_{\text{0}} = \frac{2\pi}{T_{0}} = \sqrt{\frac{g}{l}}\). 
    Donc \(\boxed{T_{0} = 2\pi\sqrt{\frac{l}{g}}}\), c'est la période propre de l'oscillateur harmonique.  
\end{eg}