\chapter{Radioactivité}
\section{la radioactivité}
\subsection{Définitions}
\begin{definition}[Radioactivité]\label{def:Radiactivité}
    La radioactivité fut découverte en 1896 par Henri Becquerel lors de ses travaux sur la phosphorescence. La radioactivité est la transformaytion spontannée de noyaux atomiques instables en d'autres atomes en émettant simultanément des particules de matière et de l'énergie.  \par 
    Le noyaux de l'atome est le siège de plusieurs interactions qui assure la cohésion des particules qui le constitue. Sous l'action de ses interactions, certains noyaux sont stables et d'autres ne le sont pas. Il existe près de 2000 noyaux d'atomes dont seulement 279 sont stables.
\end{definition}

Le diagramme (Z,N) référence l'ensemble des noyaux connus (voir doc 5 p.253)
\begin{itemize}
    \item Pour Z \(\leq\) 20, les noyaux stable se situent sur la bissectrice $Z=N$
    \item Pour Z>20, l'ensemble des noyaux stables se situent au dessus de la droite $Z=N$, donc d'avantage de neutrons que de protons.
    \item Pour Z>83, il n'existe pas de noyaux stables.
\end{itemize}

\subsection{Lois de conservation}

\begin{theorem}[Conservation]\label{thm:consmassecharge}
    Au cours d'une réaction nucléaire, il y a conservation : 
    \begin{itemize}
        \item Du nombre de charge
        \item Du nombre de masse
    \end{itemize}
\end{theorem}

\underline{Activité 2p.149}
\begin{enumerate}
    \item Les isotopes stable du plomb sont : \(^{208}_{82}Pb, ^{207}_{82}Pb, ^{206}_{82}Pb\). L'élément manquant est : \(^{209}_{82}Pb\).
    \item Voici les écritures des particules émises par la radioactivité \(\alpha =^{4}_2 He, \beta^{+} =^{0}_{1}e, \beta^{-} =^{0}_{-1}e\)
    \item On a : \(\ce{^{211}_{84}Po} = \ce{^{207}_{82}Pb} +\ce{^{4}_{2}He},\,\ce{^{207}_{83}Bi} = \ce{^{207}_{82}Pb} +\ce{^{0}_{1}e},\, \text{ et }\ce{^{207}_{81}Tl} = \ce{^{207}_{82}Pb} +\ce{^{0}_{-1}e}\) 
    \item La chaine de désintégrations se produisant à partir du plomb 210 est la suivante. Le plomb 210 émet une particule \(\beta^{-}\) et devient du bismuth 210. Celui-ci émet aussi une particule \(\beta^{-}\) et devient un Polonium 210. Celui-ci émet une particule \(\alpha\) et devient un plomb 206 : \(\ce{^{210}_{82}Pb} \to \ce{^{206}_{82}Pb} +\ce{^{4}_{2}He} + 2\ce{^{0}_{-1}e}\) 
\end{enumerate}
\newpage
\subsection{Différents types de radioactivité}
