\section{Méthodes de datation}
\subsection{Activité 4p. 151}

\begin{enumerate}
    \item On a : \( \ce{^{14}_{7}N} + \ce{^{1}_{0}n} \to \ce{^{14}_{6}C} + \ce{^{1}_{1}H}\) et \( \ce{^{14}_{6}C} \to \ce{^{14}_{7}N}+ \ce{^{0}_{-1}e}\) 
    \item \begin{enumerate}
            \item Dans un être vivant, il y a 13,56 désintégrations par minute. L'activité \(A_{0}\) est donc de $0,226$ Bq. 
            \item  \(N(t) = \frac{A(t)}{\lambda } = \frac{A(t)t_{\frac{1}{2}}}{\ln 2}\). D'ou \(N(t = 0) = 5,9 \cdot 10^{10}\).
            \item  \(M( \ce{^{}_{}C}) = 12,0 \text{g} \cdot \text{mol}^{-1}\). D'où pour \(m = 1,0\)g, on a : \(N = 8,33 \cdot 10^{-1}\text{mol} = 5,0 \cdot 10^{22}\) entités.  Ainsi, la proportion d'atomes de \( \ce{^{14}_{}C}\) est de : \(\eta  = 1,176 \cdot 10^{-12}\)  
    \end{enumerate}
    \item Avec la formule \(A(t) = \lambda N_{0} e^{ -\lambda t }\), on trouve : \(t = \frac{\ln A_{0} - \ln A(t)}{\lambda }\). On calcul donc les temps de vie \(t_{1}, t_{2}, t_{3}\) : 
            \begin{itemize}
                    \item \(t_{1} = \frac{\ln A_{0} - \ln A_{1}}{\lambda } = 489\)ans, 1520
                    \item \(t_{2} = \frac{\ln A_{0} - \ln A_{2}}{\lambda } = 7987\)ans, 2010-7987 = -5977
                    \item \(t_{3} = \frac{\ln A_{0} - \ln A_{3}}{\lambda }= 1927\)ans, 2010-1927 = 83
                    \item \(t_{4} = \frac{\ln A_{0} - \ln A_{4}}{\lambda } = 46 970\)ans    2010-46970 = -44960.
            \end{itemize}
\end{enumerate}
