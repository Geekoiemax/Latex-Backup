\section{Introduction aux ordinaux}

\begin{definition}[Ensemble]
    Un ensemble est une collection d'éléments non ordonnée et non redondante.
\end{definition}

\begin{lemma}[Axiome 1]
    On dispose de l'ensemble vide \(\emptyset\). 
\end{lemma}

\begin{lemma}[Axiome 2]
    Si on dispose d'un objet \(a\), alors \(\{a\}\) existe.
\end{lemma}

\begin{lemma}[Axiome 3]
    Si \(A,B,C, \dots\) sont des ensembles, alors \(A \cup  B \cup C \cup \dots\) est un ensemble.
\end{lemma}

\begin{corollary}[Construction des entiers.]
    \underline{Question} : Est-il possible de construire les entiers?
    \begin{itemize}
        \item 0 sera représenté par \(\emptyset \) 
        \item 1 sera représenté par \(\{0\} = \{\emptyset\}\)
        \item On réprésente alors 2 par \(\{0,1\} = \left\{ \emptyset ,\{\emptyset\}  \right\}\)  
        \item Dans le cas général, on définit \(n\) par : \(n = \{1,\dots,n-1\} = (n-1) \cup \{n-1\}\).  
    \end{itemize}
\end{corollary}

\begin{definition}[Sup]
    Soit \(X\) un ensemble de nombres (\(X = \{0,2,7\}\)) et je veux créer une fonction \(\sup\) qui à \(X\) m'associe le plus grand nombre de \(X\).
    \[
        \sup X = \bigcup_{a \in X} a 
    \] 
\end{definition}

\begin{definition}[Infini]
    On cherche \(\sup \{0,\dots,n, \dots\}\). 
    \begin{eqnarray*}
        \sup \{0,\dots\} &=& 0 \cup 1 \cup \dots \\
        &=& \{0,0,1,0,1,2, \dots\} \\
        &=& \{0,1, \dots\} \\
        &=& \omega
    \end{eqnarray*}
\end{definition}

On a ainsi : 
\begin{enumerate}
    \item \(0,1, \dots, n \) les entiers
    \item \(.^{+}\) la fonction qui à \(n\) associe son successeur
    \item \(\omega\) le supérieur de tous les entiers  
    \item \(\omega^{+} = {0,1, \dots, n, \dots, \omega}\) donc \(\omega^{+} \neq \omega\) 
\end{enumerate}

\begin{definition}[Comparaison]
    Soient \(x\) et \(y\) des ordinaux. 
    \begin{notation}
        Si \(x\) appartient à \(y\), on note \(x<y\)
    \end{notation}
    \begin{notation}
        Si \(x\) est inclus dans \(y\), on note \(x\leq y\) 
    \end{notation}
\end{definition}

\begin{corollary}[Propriétés]
    Pour tout \(n\) un ordinal on a : 
    \[
        \begin{cases}
            n < \omega \\
            n < n^{+}
        \end{cases}
    \]
    Notamment, on a : 
    \[
        \omega < \omega^{+} < \omega^{++}
    \]
\end{corollary}

\begin{corollary}[Transitivité]
    On a : 
    \[
        x<y \wedge y<z \implies x<z
    \]
\end{corollary}

\begin{definition}[Ordinal]
     \(x\) est un ordinal, si: 
    \[
        y<x \implies  y\leq x
    \]
    \[
        y<x \wedge z<x \implies 
        \begin{cases}
            y<z \\
            \text{ ou } \\
            y=z \\
            \text{ ou } \\
            y>z
        \end{cases}
    \]
\end{definition}

\begin{lemma}[Axiome de fondation]
    Soit \(x_{0}\) un ensemble, \(x_{1}<x_{0}\), \(x_{2}<x_{1}\)..., alors : 
    \[
        \exists n, \not\exists x_{n+1}
    \]   
\end{lemma}

\begin{theorem}[Minimum]
    Soit \(x\) un ordinal, \(\tilde{x} \leq x , \tilde{x} \neq \emptyset \) alors : 
    \[
        \exists y < \tilde{x} \text{ t.q. } \forall z < \tilde{x}, y \leq z
    \]
\end{theorem}

\begin{explanation}
    laissée en exercice au lecteur
\end{explanation}
\begin{theorem}[Egalité]
    Soit \(x, y \) des ordinaux : 
    \[
        x\leq y, y\geq x \implies x=y
    \]
\end{theorem}

\begin{theorem}[thm]
    Soient \(x,y\) des ordinaux, alors : 
    \[
        x\leq y \implies \begin{cases}
            x = y\\
            \text{ ou } \\
            x<y
        \end{cases}
    \]
\end{theorem}

\begin{explanation}
    Laissée en exercice au lecteur
\end{explanation}

\begin{notation}
    On note : 
    \[
        \text{"Soit x un ordinal"} \iff "\text{Soit } x \in \mathrm{On}"
    \]
    et toutes les notations que l'on peut en découler.
\end{notation}

\begin{theorem}[Construction des ordinaux]
    Soit \( x \in \mathrm{On}^{*}\) , alors l'une exactement des deux choses est vraies  :
    \begin{itemize}
        \item \(\exists y \in \mathrm{On} , x = y^{+}\) \\
        \item \(x = \sup_{a<x} a\)  
    \end{itemize} 
\end{theorem}

\begin{explanation}
    Laissée en exercice au lecteur \(\,\square\).
\end{explanation}

\begin{theorem}[BIG theorem]
    \[
        (a,b) \in \mathrm{On}^{2} \implies \begin{cases}
            a \leq b \\
            \text{ ou } \\
            b\leq a
        \end{cases}
    \]
\end{theorem}

\begin{explanation}
    Laissée en exercice au lecteur \(\,\square\).
\end{explanation}

\begin{theorem}[Le sup]
    \begin{eqnarray*}
        a \in \mathrm{On} \implies a^{+} \in \mathrm{On}\\
        X \subset \mathrm{On} \implies \sup X \in \mathrm{On}
    \end{eqnarray*}
\end{theorem}

\begin{explanation}
    Laissée en exercice au lecteur \(\,\square\).
\end{explanation}

