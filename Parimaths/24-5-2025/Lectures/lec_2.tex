\section{Opérations}

\begin{definition}[ordre]
    Soit \(E\) un ensemble quelconque. Un ordre sur \(E\) est une manière de comparer les éléments de \(E\). 
    \begin{enumerate}
        \item \(\forall x \in E, x\leq x\)\\
        \item \(\forall (x,y) \in E^{2}, x \leq y , y\geq x \implies  x = y\) \\
        \item \(\forall (x,y,z) \in E^{3}, x\leq y, y\leq z \implies x\leq z\)   
    \end{enumerate}
\end{definition}

\begin{definition}[Ordre bien fondé]
    Soit \(E\) un ensemble, un ordre \(\leq\) est bien fondé sur \(E\) ssi : 
    \[
        \forall F \subset E, \exists x \in F, x = \min F
    \] 
\end{definition}

\begin{theorem}[Isomorphisme]
    Si je prends \(E\) un ensemble ordonnée bien fondé, alors \(E\) est isomorphe à un unique ordinal.  
\end{theorem}