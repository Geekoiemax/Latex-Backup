\documentclass{article}
\usepackage[utf8]{inputenc}
\usepackage{amsmath,amssymb,amsfonts} %pour les maths
\usepackage{tikz,ifthen,tikz-3dplot} %outil pour les schemas
\usepackage{pgfplots}
\pgfplotsset{compat=1.18}

%GEOMETRY
\usepackage[a4paper,top=1in,bottom=1in,left=1in,right=1in,heightrounded]{geometry}
%line height
\renewcommand{\baselinestretch}{1.15}
%paragraphs
\setlength{\parindent}{0pt}
\setlength{\parskip}{0.8em}

\title{Problème 4450}
\author{Maxime Muller}
\date{\today}


\begin{document}
\maketitle

Montrons que \(\forall n \in \mathbb{N}, P_n : a_n \leq \frac{1}{n(n+1)}\)\\

Pour \(n=1\) : \(a_1 = \frac{1}{2}\leq \frac{1}{2}\)\\

Pour \(n=2\) : \(a_2 = \frac{1}{2}\cdot \frac{2\cdot 2-3}{2\cdot 2}\leq \frac{1}{2\cdot 3}\)\\
Pour \(n\geq 3\), On procède par récurrence, l'initialisation étant

Soit \(n\in \mathbb{N} \text{ tq } P_{n-1}\). Montrons que \(P_n\)

\begin{align*}
    a_n \leq \frac{1}{n(n+1)} &\Leftrightarrow a_{n-1} \cdot \frac{2n-3}{2n}\leq_1\\
    &\Leftrightarrow \frac{n(n+1)(2n-3)}{2n} \cdot a_{n-1} \leq 1\\
    &\Leftrightarrow \frac{(n+1)(2n-3)}{2} a_{n-1} \leq 1\\
    &\Leftrightarrow a_{n-1} \leq \frac{2}{(n+1)(2n-3)}
\end{align*}
Etudions le signe de \(\frac{1}{n(n-1)} - \frac{2}{(n+1)(2n-3)}\).\\
\begin{align*}
    \frac{1}{n(n-1)} - \frac{2}{(n+1)(2n-3)} &= \frac{(n+1)(2n-3)-2n(n-1)}{n(n-1)(2n-3)(n+1)} \\
    &= \frac{n-3}{n(n-1)(2n-3)(n+1)} \\
\end{align*}

Or : \(\forall n \in \mathbb{N}, n\geq 3, n(n-1)(2n-3)(n+1)\geq 0 \text{ et } n-3\geq 0 \)\\
D'où : \(a_{n-1} \leq \frac{1}{n(n-1)} \Leftrightarrow a_n \leq \frac{1}{n(n+1)}\)\\
Donc par récurrence, \(\forall n \in \mathbb{N}^{*}, a_n \leq \frac{1}{n(n+1)}\)

On a: \(\sum_{k=1}^{n} \frac{1}{n(n+1)} = \sum_{k=1}^{n} \frac{1}{n}-\frac{1}{n+1} = 1-\frac{1}{n+1}\)
D'où \(\sum_{k=1}^{n} a_n \leq \sum_{k=1}^{n} \frac{1}{n(n+1)}\)\\
\(\Leftrightarrow \sum_{k=1}^{n} a_n \leq 1 - \frac{1}{n+1} < 1\)
\end{document}