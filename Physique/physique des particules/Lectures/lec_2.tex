\begin{definition}[Désintégration \(\alpha\) ]\label{def:arad}
    Les noyaux lourds subissent un désintégration \(\alpha \). Ils se désintègre en émétant un noyaux d'hélium : 
    \[
        \ce{^{A}_{Z}X} \to \ce{^{A-2}_{Z-2}Y} + \ce{^{4}_{2}\alpha }
    \] 
\end{definition}

\begin{definition}[Désintégration \(\beta^{-}\)]\label{def:bmrad}
    Elle concerne les noyaux instables situés dans la zone rouge du document 5 p. 153. Ils contiennent trop de neutrons. 
\[
    \ce{^{A}_{Z}X} \to \ce{^{A}_{Z+1}Y} + \ce{^{0}_{-1}e}
\]

Dans le noyau, un neutron c'est transformé en un proton et un électron. 
\end{definition}

\begin{definition}[Désintégration \(\beta^{+}\) ]\label{def:bprad}
    Elle concerne les noyaux instables situés dans la zone bleu du document 5 p. 153. Ils contiennent trop de protons. 
    \[
        \ce{^{A}_{Z}X} \to \ce{^{A}_{Z-1}Y} + \ce{^{0}_{1}e}
    \]
    Dans le noyau, un proton c'est transformé en un neutron et un positron. 
\end{definition}

\begin{definition}[Désexcitation \(\gamma\) ]\label{def:grad}
    Les noyaux issus de réactions de désintégration sont en général obtenu dans un état excité. Le retour a l'état fondamental s'accompagne de l'émission d'un rayonnement \(\gamma\) de très courte longueur d'onde et très pénétrante. 
\end{definition}

\subsection{Bilan}
%insérer schema%
\section{Loi de décroissance radioactive}

\begin{definition}[Activité d'un échantillon radioactif]\label{def:radact}
    L'activité d'un échantillon est égale au nombre de désintégration qui se produisent par unité de temps dans cet échantillon : 
\[
    A(t) = -\frac{dN(t)}{dt}
\]
Avec \(A(t)\) l'activité en becquerel (1Bq = 1 désintégrations par seconde), et \(N(t)\) le nombre de noyaux radioactifs à l'instant t.
\end{definition}

\begin{definition}[Constante radioactive]\label{def:radcste}
    L'activité d'un échantillon est proportionel au nombres de noyaux qu'il contient et dépend également du type de noyau. 
    \[
        A(t) = \lambda N(t) \text{ avec $\lambda$ la constante radioactive en s$^{-1}$}
    \]
\end{definition}

\subsection{Décroissance exponentielle }
On aboutit à l'équation différentielle suivante : 
\[
    \frac{d}{dt}\left[ N(t) \right] + \lambda N(t) = 0
\]
La solution générale est : \(N(t) =  \mu e^{-\lambda t}\). En notant la condition initale \(N(t =0) = N_{0}\), on obtient :
\[
    N(t) = N_{0}e^{-\lambda t}
\]
\pagebreak

\subsection{Temps de demi vie radioactive}
\begin{definition}[Temps de demi vie radioactive]
    C'est la durée nécessaire pour que la moitié des noyaux radioactifs initialement présents se soient désintégrés (doc 12 et 13 p. 156)
\end{definition}

On a donc : 
\begin{eqnarray*}
    \frac{N_{0}}{2} &=& N_{0}e^{-\lambda t_{\frac{1}{2}}} \\
    -\ln 2 &=& -\lambda t_{\frac{1}{2}}\\
    t_{\frac{1}{2}} &=& \frac{\ln 2}{\lambda } 
\end{eqnarray*}
