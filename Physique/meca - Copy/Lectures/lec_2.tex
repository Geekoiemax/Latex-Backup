\section{Description de l'écoulement d'un fluide}
\subsection{Definition et vocabulaire}
\subsubsection{Description du mouvement}

Un fluide n'étant pas pas un système ponctuel, il faut se poser la question de ce que l'on étudie exactement.\\
\begin{definition}[Description Lagrangienne du fluide]\label{def:desclag}
    Etude du fluide en étudiant l'évolution de petites parties au cours du temps.\\
    Cela équivaut à "marquer" une "particule fluide" en utilisant un indicateur coloré : 
\end{definition}

\begin{remark}[pratique]
    Dans la pratique, il est difficile d'identifier et donc de suivre une "particule fluide" en mouvement. Il apparait donc judicieux d'introduire une description alternative pour le mouvement. 
\end{remark}

\begin{definition}[Description eulérienne du mouvement]\label{def:desceul}
    On étudie donc des champs de l'ensemble du fluide (champ de vitesse à un instant t donné par exemple). C'est la description eulérienne du fluide.\\ 
    Ainsi on est amené à étudier des champs de grandeur cinématique (position, vitesse, accélération) et des champs de grandeur thermodynamiques (masse volumique, température, pression).
\end{definition}
\begin{definition}[Fluide incompressible]\label{def:incompressible}
    Un fluide est dit incompressible lorsque \(\rho  = \text{ cste }\)sous l'action d'une pressions externe.
    \[
        \frac{d \rho }{dt} = 0
    \]
    Dans la pratique, la compressibilité des liquides et des solides est quasi-nulle (pour l'eau, c'est 1 pour 1000). On les assimile donc généralement comme incompressibles, ce qui n'est pas vrai pour les gaz.
\end{definition}

\begin{definition}[Fluide parfait]\label{def:fparf}
    Un fluide est dit parfait lorsque les forces de frottement au sein du fluide sont négligeables. Sa viscosité est nul. Ce n'est pas le cas de tous les fluides : par exemple le miel.
\end{definition}

\begin{definition}[Ecoulement permanent]\label{def:ecoulperm}
    En régime permanent, les grandeurs cinématiques et thermodynamiques ne dépendent pas du temps. Elles dépendent alors seulement de l'espace.
\end{definition}